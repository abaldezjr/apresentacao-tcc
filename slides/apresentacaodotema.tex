\subsection{Apresentação do tema}

%SLIDE DE APRESENTAÇÃO DO TEMA
\begin{frame}
\frametitle{Apresentação do tema}
\begin{itemize}
    \item A mecânica dos fluidos é uma área muito complexa da engenharia.
    \item Nem sempre é possível  projetar com precisão sem uma análise prévia da ação de esforços sobre algum material. 
    \item O estudo da ação do ar sobre estruturas pode ser um fator determinante para o sucesso de um projeto. 
\end{itemize}
\end{frame}

%SLIDE DE APRESENTAÇÃO DO TEMA
\begin{frame}
\frametitle{Apresentação do tema}
\begin{itemize}
    \item A análise aerodinâmica pode apresentar dados confiáveis ao projetista para tomada de decisão. 
    \item Através das leis de similaridade, aplica-se fatores de escala para replicar os resultados em escalas reais. 
    \item De uma forma menos onerosa é possível se fazer esse estudo em escala reduzida e com condições controladas em laboratório.
\end{itemize}
\end{frame}

%SLIDE DE TÚNEL DE VENTO
\begin{frame}
\frametitle{Túnel de vento}

Os túneis de vento são as bancadas de testes para estudos de escoamento de ar, onde é possível simular cenários e avaliar a interação do fluido e estrutura.

\begin{figure}
\centering
\includegraphics[scale = 0.6]{figuras/tuneldevento}
\end{figure}

\end{frame}
 
% SLIDE DOS INSTRUMENTOS DE MEDIÇÃO
\begin{frame}
\frametitle{Instrumentos de medição}

Uma das grandezas obtidas é a velocidade do fluido, a qual é medida por instrumentos como tubo de Pitot ou sonda de anemômetro de fio quente.

    \begin{columns}
        \column{0.5\textwidth}
            \begin{figure}
            \centering
            \includegraphics[scale = 0.6]{figuras/tubodepitotaviao}
            \end{figure}
        \column{0.5\textwidth}
            \begin{figure}
            \centering
            \includegraphics[scale = 0.6]{figuras/sonda}
            \end{figure}
    \end{columns}
\end{frame}
